%=========================================
\subsection{Минимальные параметры технических средств}
Для оптимальной работы приложения необходимо учесть следующие системные требования:
\begin{my_enumerate}
\item Тонкий клиент Orange Pi Lite, оснащенный:
    \begin{my_enumerate}
    \item Обязательно процессор Allwinner H3 с тактовой частотой 1 гигагерц (ГГц) или выше;
    \item 0.5 ГБ оперативной памяти (ОЗУ);
    \item 0.5 ГБ свободного места на жестком диске;
    \item Периферия: выход GPIO типа Rasberry Pi B+
    \end{my_enumerate}
\item Опционально: Компьютер для удаленного доступа к Orange Pi Lite, оснащенный:
    \begin{my_enumerate}
    \item Обязательно 64-разрядный (x64) процессор с тактовой частотой 1 гигагерц (ГГц) или выше;
    \item 1 ГБ оперативной памяти (ОЗУ);
    \item 1.5 ГБ свободного места на жестком диске;
    \item Wifi модулем (если Orange Pi Lite подключен к сети, то можно воспользоваться и стандартным Ethernet портом) или TTL переходником для подключения к тонкому клиенту Orange Pi Lite.
    \end{my_enumerate}
\item Монитор
\item Мышь
\item Клавиатура
\end{my_enumerate}


%=========================================
\subsection{Минимальные программные средства}
Исходный код программы для контролирования программатора обязательно должен быть написан с использованием языка C. Приложению необходим тонкий клиент с операционной системой производной от Debian с версией ядра не ниже 3.1. Приложение можно запускать как с  самого Orange Pi Lite, так и с компьютера имеющего удаленный доступ к Orange Pi Lite. В данном случае на удаленном компьютере и на тонком клиенте должны быть установленны программы для настроек удаленного доступа (например SSH, VNC viewer, TTY serial console). Для данных програм подходит любой дистрибутив Linux или 64-битная операционная система Windows 7 или более поздняя версия Windows.


\subsection{Численность и калификация персонала}
Минимальное количество персонала, требуемого для работы программы: 1 оператор. Пользователь программы должен иметь образование не ниже среднего, обладать практическими навыками работы с компьютером и электронникой.
