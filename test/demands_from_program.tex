

%=========================================
\subsection{Требования к функциональным характеристикам}
\subsubsection{Требования к составу выполняемых функций}
\begin{my_enumerate}
\item Чтение данных из формата INTEL HEX8M для хранения программы прошивки.
\item Возможность отдельной записи EEPROM памяти, не стирая програмную память микроконтроллера.
\item Поддержка 3 линеек микроконтроллеров серии 16F: 627A / 628A / 648A.
\item Проверка входного файла на корректность.
\item Графический интерфейс для оперирования программой.
\item Интерфейс командной строки для оперирования программой.
\item Повышаюший переходник с 3.3В на 5В для взаимодействия с микроконтроллером.
\item Схемотехника для платы которая позволяет подключить микроконтроллер к тонкому клиенту Orange Pi Lite.
\item Завершенные, работающие схемы на макетной плате.
\item Схемы разводки макетной платы для подключения микроконтроллера к Orange Pi Lite. 
\end{my_enumerate}

\subsubsection{Требования к организации входных и выходных данных}
Входными данными для работы программатора являются скомпилированный файл программы, микроконтроллер подключенный к плате, а также (для обеспечения взаимодействия с пользователем) клавиатура и/или мышь. Входной файл данных может быть созданн в любой среде разработки и любым компилятором поддершивающим формат INTEL HEX8M. Примером такой среды разработки является MPLAB X (https://www.microchip.com/, разработчик: коммерческая организация Microchip Ltd.).

\begin{my_enumerate}
\item Из-за огромного количества серий микроконтроллеров поддерживать их все не представляется возможным. Поэтому программа должна работать только с микроконтроллерами PIC серии 16F, конкретно с линейками 627A / 628A / 648A.
\item Файл программы должен соответствовать формату INTEL HEX8M. По сравнению с двумя другими часто встречающимися форматами INTEL HEX8S, INTEL HEХ32, данный формат наиболее оптимально подходит под серию 16F. В силу того что память 14-битных микроконтроллеров не превышает 64 килобайт (здесь подходит формат HEX32) и програмное слово не нуждаеться в разбиении на высокий и низкий байт как в 16-битных микроконтроллерах (здесь подходит формат HEX8S).
\item Пользователь должен иметь возможность модифицировать следующие входные данные в процессе работы программы в усливиях графического интерфейса и перед запуском программы в командной строке:
\begin{my_enumerate}
\item Указать что требуется запись EEPROM памети без модификации програмной памяти микроконтроллера.
\item Указать что требуется проверить входной файл на ошибки.
\item Указать что требуется записать входной файл в програмную память и в EEPROM память микроконтроллера.
\item Поменять уровень колличества сообщений выводимих программой пользователю.
\item Отменить процесс программирования.
\end{my_enumerate}
\end{my_enumerate}

\medskip
Выходными данными для программатора является запрограммированный микроконтроллер, данные на экране и индикатор программирования на плате программатора.


\subsubsection{Прочие требования}
\begin{enumerate}
\item Поддержка изменения размеров окна.
\item Использование Qt для создания интерфейса.
\end{enumerate}

%=========================================
\subsection{Требования к временным характеристикам}
\begin{enumerate}
\item Задержка между сигналом к началом программирования не должна быть меньше чем 0.001 секунда и не должна превышать 0.1 секунд для файлов программ размером меньше чем 5 килобайт.
\end{enumerate}


%=========================================
\subsection{Требования к интерфейсу}
Интерфейс должен быть прост в использовании. Он должен предоставлять возможность
\begin{my_enumerate}
\item Прочитать данные из формата INTEL HEX8M для хранения программы прошивки.
\item Возможность запрограммировать EEPROM память, не стирая програмную память микроконтроллера.
\item Проверить входной файл на корректность.
\end{my_enumerate}

Командный интерфейс должен предоставлять те же возможности что и графический интерфейс и следовать стандартам принятым при создании интерфейсом командной строки в системе Linux.

%=========================================
\subsection{Требования к надежности}
\subsubsection{Обеспечение устойчивого функционирования программы}
Программа не должна вне зависимости от входных данных или действий оператора завершатся аварийно. При некорректно введенных параметрах пользователю должно отображаться сообщение об ошибке.
\subsubsection{Время восстановления после отказа}
Требования к восстановлению после отказа не предъявляются.
\subsubsection{Отказы из-за некорректных действий оператора}
В случае открытия файла, не соответствующему требованиям ко входным данным, пользователю должно отображаться сообщение об ошибке.
