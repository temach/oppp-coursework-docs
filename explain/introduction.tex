\subsection{Наименование}
Наименование: «Программатор микроконтроллеров PIC на основе Orange PI Lite». \\
Наименование на английском: «Programmer for PIC Microcontrollers Based on Orange PI Lite». \\


\subsection{Краткая характеристика}
    Цель работы - реализовать программатор для микроконтроллеров PIC серии 16F на тонком клиенте Orange PI Lite.
    В задачи работы входит расчет и инженерия электронной схемы для программирования, написание программы для управления этой схемой.
    Электронная схема предоставляет возможность подключить микроконтроллер PIC серии 16F к тонкому клиенту Orange Pi Lite, управлять уровнями вольтажа на 5В и на 3В и возможность проверить процесс программирования на светодиодах.         
    Программа предоставляет пользователю командный и графичексий интерфейсы, чтение файлов INTEL HЕХ8М, возможность записать файлы программы в программную и EEPROM память микроконтроллера.
    В состав работы также входит создание демонстрационных исходных данных (файлов) для данного программатора и микроконтроллеров серии 16F.

\smallskip
Файл программы в формате INTEL HEX8M, удовлетворяющий требованиям входных данных, может быть получен в результате компиляции исходного кода одним из компиляторов для микроконтроллеров серии PIC 16F. Обычно для разработки используются пакеты предоставляющие интегрированную среду разработки. Например пакет MPLAB X (https://www.microchip.com/, разработчик: организация Microchip Ltd.)


\subsection{Документы, на основании которых ведется разработка}
Разработка программы ведется на основании приказа 
\textnumero 6.18.1-02/1112-19 от 11.12.2016
«Об  утверждении  тем,  руководителей  курсовых  работ  студентов
образовательной  программы  Программная  инженерия 
факультета 
компьютерных наук» в соответствии с учебным планом подготовки бакалавров по направлению «Программная инженерия», факультета Компьютерных наук,
Национального исследовательского университета «Высшая школа экономики» 