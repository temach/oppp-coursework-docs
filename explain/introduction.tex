\subsection{Наименование}
Наименование: «Программа скелетная анимация». \\
Наименование на английском: «Program of Skeletal Animation». \\


\subsection{Краткая характеристика}
    Цель работы - реализовать программу скелетной анимации.
    В задачи работы входит загрузка анимации из файлов collada (.dae), расчет промежуточных кадров анимации и воспроизведение анимации на экране средствами OpenGL.
    Также программа предоставляет пользователю возможность менять положение камеры, перейти к любому моменту времени в анимации и просмотреть иерархию костей и модели.
В состав работ также входит создание демонстрационных исходных данных (файлов) для данной программы.

\smallskip
Файл анимации в формате collada, удовлетворяющий требованиям входных данных, может быт подготовлен пользователем в любом пакете для трех мерного моделирования. Например в программе Blender (https://www.blender.org/, разработчик: некоммерческая организация Blender Foundation)


\subsection{Документы, на основании которых ведется разработка}
Разработка программы ведется на основании приказа 
\textnumero 6.18.1-02/1112-19 от 11.12.2015 
«Об  утверждении  тем,  руководителей  курсовых  работ  студентов
образовательной  программы  Программная  инженерия 
факультета 
компьютерных наук» в соответствии с учебным планом подготовки бакалавров по направлению «Программная инженерия», факультета Компьютерных наук,
Национального исследовательского университета «Высшая школа экономики» 