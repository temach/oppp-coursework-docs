\subsection{Оринтировочная экономическая эффективность}
Оринтировочная экономическая эффективность не рассчитывается.

\subsection{Экономические преимущества разработки}
Существующими коммерчискими аналогами данного приложения являются серийные программаторы комpaнии Microchip Ltd. В силу того что схематехника и код данного программатора распростроняется бесплатно, экономически выгодным аналогом к нему являются известный программатор K150 (подключающийся к компьютеру через USB), а также разнообразные схемы основанные на портах LDP (для принтера) и COM портах. Однако схематехника K150 в несколько раз сложнее, к тому же для его работы требуются два уже заранее запрограммированных микроконтроллера, поэтому собрать его в домашних условиях невозможно (требуется приобрести 2 уже запрограммированных микроконтроллера). Варианты основанные на портах LDP и COM также не подходят поскольку эти потры морально устарели и все реже и реже присутствую на современных компьютерах. Данная разработка позволяет использовать совреммый и актуальный тонкий клиент Orange Pi Lite в качестве программатора микроконтроллеров, к тому же все части для сборки программатора стоят гораздо дешевле чем покупка официального прибора от Microchip, и наконец они предельно просты и позволяют сборку программатора в домашних условиях.