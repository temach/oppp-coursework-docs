\subsection{Формат INTEL HEX8M (.hex)}
INTEL HEX8M — это формат, разработанный для обмена скомпилированным кодом но не в бинарном а в текстовом формате.
В формате указывается связь между адресом и байтовым значением, которое должно находиться по этому адресу.

Формат включает в себя подсчет проверочной суммы, котоая записывается последним байтом в конце строки.

Ниже приведен пример описания простой прогаммы для мигания светодиодом на микроконтроллере PIC16F628A в данном формате:

\begin{small}
\begin{verbatim}
:1000000000000000000000000000000000000000F0
:0400100000000000EC
:100032000000280040006800A800E800C80028016D
:100042006801A9018901EA01280208026A02BF02C5
:10005200E002E80228036803BF03E803C8030804B8
:1000620008040804030443050306E807E807FF0839
:06007200FF08FF08190A57
:00000001FF
\end{verbatim}
\end{small}
