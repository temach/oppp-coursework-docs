\subsection{Терминология}
\begin{description}

\item[Корневая вершина (англ. root node)]  
Самый верхний узел дерева.

\item[Полигональная сетка (жарг. меш от англ. polygon mesh)]
Совокупность вершин, рёбер и граней, которые определяют форму многогранного объекта в трехмерной компьютерной графике и объёмном моделировании. Гранями являются треугольники.

\item[Дерево]
Связный ациклический граф. Связность означает наличие путей между любой парой вершин, ацикличность — отсутствие циклов и то, что между парами вершин имеется только по одному пути.

\item[Степень вершины]
Количество инцидентных ей (входящих/исходящих из нее) ребер.

\item[Интерполяция, интерполирование анимации]
Способ нахождения промежуточных значений состояния анимации по имеющемуся дискретному набору известных значений.

\item[Z-буферизация]
В компьютерной трёхмерной графике способ учёта удалённости элемента изображения. Представляет собой один из вариантов решения «проблемы видимости»

\item[Z-конфликт (англ. Z–fighting)]
Если два объекта имеют близкую Z-координату, иногда, в зависимости от точки обзора, показывается то один, то другой, то оба полосатым узором.

\item[OpenGL (Open Graphics Library)]
Спецификация, определяющая независимый от языка программирования платформонезависимый программный интерфейс для написания приложений, использующих двумерную и трёхмерную компьютерную графику. На платформе Windows конкурирует с Direct3D.

\item[Рендеринг (англ. rendering — «визуализация»)]
Термин в компьютерной графике, обозначающий процесс получения изображения по модели с помощью компьютерной программы.

\item[Текстура]
Растровое изображение, накладываемое на поверхность полигональной модели для придания ей цвета, окраски или иллюзии рельефа. Приблизительно использование текстур можно легко представить как рисунок на поверхности скульптурного изображения.

\end{description}

