\subsection{Терминология}
\begin{description}

\item[EEPROM]  
электрически стираемое перепрограммируемое ПЗУ (ЭСППЗУ), один из видов энергонезависимой памяти (таких, как PROM и EPROM). Память такого типа может стираться и заполняться данными до миллиона раз.

\item[Архитектура набора команд (англ. instruction set architecture, ISA)]
часть архитектуры компьютера, определяющая программируемую часть ядра микропроцессора. На этом уровне определяются реализованные в микропроцессоре конкретного типа

\item[Язык ассемблера (англ. assembly language)]
машинно-ориентированный язык низкого уровня с командами, не всегда соответствующими командам машины, который может обеспечить дополнительные возможности вроде макрокоманд.

\item[MPLAB]
интегрированная среда разработки, представляющая собой набор программных продуктов, предназначенная для облегчения процесса создания, редактирования и отладки программ для микроконтроллеров семейства PIC, производимых компанией Microchip Technology. Среда разработки состоит из отдельных приложений, связанных друг с другом и включает в себя компилятор с языка ассемблер, текстовый редактор, программный симулятор и средства работы над проектами, также среда позволяет использовать компилятор с языка C.

\item[контрольный таймер, англ. Watchdog timer]
аппаратно реализованная схема контроля над зависанием системы. Представляет собой таймер, который периодически сбрасывается контролируемой системой. Если сброса не произошло в течение некоторого интервала времени, происходит принудительная перезагрузка системы. В некоторых случаях сторожевой таймер может посылать системе сигнал на перезагрузку («мягкая» перезагрузка), в других же — перезагрузка происходит аппаратно (замыканием сигнального провода RST или подобного ему).

\item[Внутрисхемное программирование (англ. In-System Programming, сокр. ISP)]
технология программирования электронных компонентов (ПЛИС, микроконтроллеры и т. п.), позволяющая программировать компонент, уже установленный в устройство. До появления этой технологии компоненты программировались перед установкой в устройство, для их перепрограммирования требовалось их извлечение из устройства.
 
\item[Универсальный асинхронный приёмопередатчик (англ. UART)]
узел вычислительных устройств, предназначенный для организации связи с другими цифровыми устройствами. Преобразует передаваемые данные в последовательный вид так, чтобы было возможно передать их по одной физической цифровой линии другому аналогичному устройству. Метод преобразования хорошо стандартизован и широко применяется в компьютерной технике (особенно во встраиваемых устройствах и системах на кристалле (SoC)).


\end{description}

